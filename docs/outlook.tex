% !TeX root = ./corona_contact_tracing.tex
% chktex-file 46
% !TeX spellcheck = en-GB
% !TeX encoding = utf8

\section{Outlook}
This section presents future approaches to optimally reduce the spread of the disease with minimal societal intervention.

\subsection{Edge Cutting}
All non pharmaceutical interventions (NPI) could be understood as some kind of edge removal in our graph-based approach:
\begin{itemize}
	\item Isolation of an infected individual removes all of its edges with very high probability.
	\item Quarantine of a contact person removes all of its edges with high probability.
	\item Social distancing removes some edges of of many individuals.
	\item Cancellation of large events remove many edges of many individuals.
\end{itemize}
These interventions can be modelled in our SIR model by modifying the diffusion rate of the agents.
Thus its effects can be evaluate qualitatively and quantitatively.

Based on our graph-based approach, one could also try to compute the minimal set of interventions, which still contains the disease.
A good proxy for this containment is to analyse the effect of these measures on $R_0$, which models how many people are infected by a single infected person.
It is crucial to constrain $R_0$ below one to avoid exponential growth of the number of infected individuals.

In our approach $R_0$ can be derived from the ratio of infected people of $H^{(t+1)}$ and of $H^{(t)}$:
\begin{equation}
	R_0 = \frac{\norm{H^{(t+1)}\mid_{m=1}}_{0}}{\norm{H^{(t)}\mid_{m=1}}_{0}} %#TODO betrag
\end{equation}

The square matrix $C$ with dimensions $N \times N$ models desirable edge cancellations.
Note, that this matrix does not have to know the edges of a future time step, it only expresses which edges must not exist.
It is multiplied element-wise onto the adjacency matrix $A$, thus $\bar{A} = A \odot C$ describes a adjacency matrix with applied cancellations.

To optimally limit the spread of the disease, we seek to minimize the number of cancellations $\norm{C}_0$, given that $R_0$ is below one:
\begin{equation}
	\max_{C} \norm{C}_0\text{, s.t. }R_0 < 1
\end{equation}
Alternatively

\begin{equation}
	\max_{C} \norm{C}_0\text{, s.t. } \norm{H^{(t+1)}\mid_{m=1}}_{0} <= KapaLimit
\end{equation}

% Future Work: This would be more powerful, if there would be some different kinds of edges (social, work, education, large events, etc.)\\
% Future Work: This only takes the current time step into account but it would be desirable to look even further into the future.

\subsection{Test Prioritization}
When tests are limited, they should be used to discover as much as possible about the health state of the overall population.
This in turn allows to reduce the $R_0$ value in further time steps as edge cutting becomes more efficient.

Lets assume there are $t_{\text{max}}$ tests per time step.
A test reveals the true health state of an individual (ignoring false negatives and false positives)
\begin{equation}
	h_{{v}_i}^{(t)} \xrightarrow{\text{test}} h_{{v}_i}^{(t+1)} \in \{\vec{e}_0, \vec{e}_1, \vec{e}_2 \}
\end{equation}

The test assignment $T$ with dimension $N$ is a binary variable describing which individuals should be tested.