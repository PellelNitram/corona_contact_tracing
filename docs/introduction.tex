% !TeX root = ./corona_contact_tracing.tex
% chktex-file 46
% !TeX spellcheck = en-GB
% !TeX encoding = utf8

\section{Introduction}
The outbreak of the SARS-COV-2 virus and the associated COVID-19 illness sweeps rapidly across the world.
Some patients need ventilation support to survive and the exponential growth in infected persons quickly overwhelms any available medical resources.\\
Thus the identification of contact persons is of great importance to control the spread of the disease.
Some governments such as Singapurian use location trace data from mobile phone providers.
Other approaches are more user centric and build apps, which utilize GPS data of individuals.

It is known from past outbreaks and epidemilogic research that such contact tracing and non pharmaceutical interventions (NPI) like school cancellations are important tools. 
But both are not very directed interventions and do not reveal anything about the health state of the entire population.

The following section will first explain the mathematical foundations of our SIR model to generate realistic ground truth data.
Afterwards our novel graph-based approach to probabilistic health prediction of the entire population will be explained.
In the outlook we sketch how this approach can be utilized to measure the effect of contact tracing and NPIs.
Furthermore, we propose to use mathematical optimization to compute very targeted NPIs and optimal placement of limited tests.