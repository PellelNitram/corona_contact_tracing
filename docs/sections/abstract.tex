\begin{abstract}
	
	\begin{description}
		\item[Scope] Created during the \#WirVSVirus Hackathon of the German government as initiative against the spreading of the COVID-19 pandemic in early 2020.
		\item[Open source] Available under \url{https://github.com/PellelNitram/corona_contact_tracing}.
	\end{description}

	% \todo[inline]{Finalise abstract as last part.}
	
	In this work we develop a graph convolutional approach to predict the health status of all agents in this simulation.	It uses past contacts as well as observed health information to derive this prediction.
	It is able to deal with partial data such as missing locations and missing health status, as it is a probabilistic approach.
	
	Lastly, this prediction can be used to derive measures to reduce the diseases' spread.
	We propose to find a optimal trade of between removing the least amount of edges in said graph (e.g.\ through quarantine, social distancing, etc.) and limiting the spread of the disease.
	% Such trade of could be found with minimal cut algorithms.
	Opposed to classical non-pharmaceutical intervention methods such as contact tracing, our approach directly identifies the nodes with the greatest potential to accelerate the diseases' spread in the network.
	
	This technical report was created within the \#WirVsVirus Hackathon of the German government and is Work in Progress!

\end{abstract}