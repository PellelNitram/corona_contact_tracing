- novel graph-based approach to

\subsubsection{Explanation of the graph contribution term}
The graph contribution models how infected agents spread the disease through contacts with susceptible agents.

\begin{equation}
h_{v_i, m}^{(l+1)}
=
\underbrace{
	\sum_k \textcolor{red}{\frac{\hat{A}_{v_i, k}^{(l)}}{\sum_j \hat{A}_{v_i, j}^{(l)}}} h_{k, m}^{(l)} \textcolor{blue}{\delta_{m, e_I}}
}_{\text{Graph}}
\end{equation}

\begin{itemize}
	\item A sum over all agents' features $h_{k,m}^{(l)}$ is weighted by the normalised infection-adjusted graph connections as shown in red.
	\item The Kronecker delta, as shown in blue, ensures that only the I feature is added as this is the only one that matters during social contacts between agents.
	\item The infection-adjusted adjacency matrix $\hat{A}$ is constructed from $A$ and $I$ which are the regular continuous adjacency matrix and the infection matrix, respectively. These three quantities are explained in the following:
	\begin{itemize}
		\item The adjacency matrix $A$ is time dependent, $A^{(l)}$, and inferred from data. In our use case, $A_{ij} = \frac{1}{dist(v_i, v_j)+\epsilon}$, hence $A_{ij}$ is large when persons $i$ and $j$ have been in contact. $\epsilon$ serves as regularization for small distances.
		\item The infection matrix is constructed as
		\begin{equation}
		I =
		\begin{pmatrix}
		0     &  0  & 0 \\
		\beta &  0  & \alpha \\
		0     &  0  & 0
		\end{pmatrix}
		=
		(I_{ij})_{i,j}
		\end{equation}
		with $i$ as the index of the host state and $j$ is the index of the contact person state. The states that we consider here are ordered as follows: susceptible, infected, recovered. $\beta$ denotes the probability of infection  after contact (also known as attack rate). $\alpha$ models the probability of being reinfected, which we assume to be zero ($\alpha=0$) based upon current medical \todo{cite!}research.
		\item $\hat{A}$, with $\hat{A}_{ij}\in [0, 1]$, is the adjacency matrix that takes the infection interactions into account and is computed as follows
		\begin{equation}
		\hat{A}_{ij} = A_{ij}\cdot \frac{ h_{v_1}^T I h_{v_2} + h_{v_2}^T I h_{v_1} }{\beta}.
		\end{equation}
		The weighted scalar product of the health states of agents $i$ and $j$ is used to evaluate whether the edge is relevant for the infection dynamics. Only when an infected person and a susceptible have contact, the edge $A_{ij}$ should be considered, otherwise it should be dropped.	The sum in the denominator comes from the fact that both, agent $i$ and $j$, can act as host during a contact. The division by $\beta$ normalises the factor to one to ensure $\hat{A}_{ij} \in [0, 1]$. Since $I$ is not symmetric, $p_a$ is a proper normalization because the sum is in $\{0, p_a\}$. Note that the fraction has the desired properties for pure $S$-, $I$- and $R$-persons.
	\end{itemize}
\end{itemize}


\subsubsection{Temporal}
% \item Explanation of the temporal term:
The transition of a persons' health state $h_{v_i}^{(l)}$ is determined by the following assumptions:
\begin{itemize}
	\item A susceptible person always stays susceptible
	\item An infected person has a probability $\gamma$, called recovery rate, to recover. The remaining probability $1-\gamma$ denotes that the person stays sick.
	\item A recovered person could have a probability to be re-infected, but we assume this to be zero. Thus a recovered person always stays recovered.
\end{itemize}
Thus the temporal transition matrix $T$ is:
\begin{equation}
T = 
\begin{pmatrix}
1 &     0    & 0      \\
0 & 1-\gamma & \gamma \\
0 &     0    & 1      \\
\end{pmatrix}
\end{equation}
The temporal update rule based on the health status thus becomes:
\begin{equation}
H^{(l+1)} = H^{(l)} T
\end{equation}

\subsubsection{Joint Model}

Our main propagation rule is based on the definition of a graph convolution shown in Eq.~\eqref{eq:graph_convolution} and reads as follows in component notation
\begin{equation}
h_{v_i, m}^{(l+1)}
=
\underbrace{
	\sum_k \textcolor{red}{\frac{\hat{A}_{v_i, k}^{(l)}}{\sum_j \hat{A}_{v_i, j}^{(l)}}} h_{k, m}^{(l)} \textcolor{blue}{\delta_{m, e_I}}
}_{\text{Graph}}
+
\underbrace{
	{(h_{v_i}^{(l)}\cdot T)}_m
}_{\text{Temporal}}
\end{equation}
with $m$ being the index of the health state.

This propagation rule is based on the following aspects:

\begin{itemize}
	\item The propagation consists of two parts, first the graph contribution and second the temporal contribution. While the former captures the dynamics of infections based on the social contacts between agents, the former ensures that an infected agent heals over time and becomes resistant against the Corona virus.
\end{itemize}