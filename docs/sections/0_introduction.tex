The outbreak of the SARS-COV-2 virus and the associated COVID-19 illness sweep rapidly across the world. Some patients need ventilation support to survive and the exponential growth in infected persons quickly overwhelms any available medical resources~\cite{10.1001/jama.2020.2648}.

Thus the identification of contact persons is of great importance to control the spread of the disease. Some governments, such as the Singapurian, use location trace data from mobile phone providers. Other approaches are more user centric and build apps that utilise GPS data of individuals.

It is known from past outbreaks and epidemiologic research that such contact tracing and non pharmaceutical interventions (NPI) like school cancellations are important tools to reduce the impact of crisis like the ongoing one. However, both are not particularly directed interventions might come at the cost of an increased social or economic cost.

We formulate a mathematical framework that is suitable to be used for subsequent optimisations which contacts should be avoided while on the other hand decreasing the social and economic cost.

The section \ref{sec:basics} will explain the mathematical foundations that are necessary to understand our main approach presented in section \ref{sec:framework}. Sections \ref{sec:working_state} and \ref{sec:outlook} present our working state at the end of the \#WirVsVirus hackathon and provide extensive outlook, respectively. In the outlooks, additionally, we propose to use mathematical optimisation to compute very targeted NPIs (e.g.\ only cancellation of large events) and optimal placement of limited tests (e.g.\ prioritize potential super spreaders). These perspectives have the potential to control the disease with minimal effects on daily life.