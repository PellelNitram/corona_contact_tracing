- next steps

- re-model current strategies: doing nothing, social distancing, isolation.

This section presents future approaches to optimally reduce the spread of the disease with minimal societal intervention.


\subsection{Policy Design}
In the previous sections we have shown how predict the health state of all individuals in a location tracked population.
The next step is to use this information to compute optimal policies.
These policies guide a governments' and societies' response to the spread of COVID-19 and modify how the disease is able to spread in a population.

An optimal policy always keeps the infection counts below the medical systems' capacity and the total number of infections as small as possible.
% not much real data on this.
Mathematically speaking we want to minimize the sum of all future infections
\begin{equation}
	\min \sum_{\forall t} N^{(t)}_i
\end{equation}
whilst
\begin{equation}
	N^{(l)}_i \leq N_{limit}, \forall t
\end{equation}

A policy may consist out of two different kinds of actions, non pharmaceutical interventions (NPI) and test prioritization (TP).


\subsubsection{Non Pharmaceutical Interventions (NPI)}
All non pharmaceutical interventions can be understood as some kind of edge removal in our graph-based approach:
\begin{itemize}
	\item Isolation of an infected individual removes all of its edges with very high probability.
	\item Quarantine of a contact person removes all of its edges with high probability.
	\item Social distancing removes some edges of many individuals.
	\item Cancellation of large events remove many edges of many individuals.
\end{itemize}

Formally speaking, the square matrix $C \in \mathbb{B}$ with dimensions $N \times N$ models desirable edge cancellations.
This is the policy, which will be optimized.
To avoid the trivial solution of $C=0$, the cancellation of all edges, we also want to minimize the the number of cancellations
\begin{equation}
	\min_{C} -\sum_i \sum_j C_{ij}
\end{equation}


Note, that this matrix does not have to know the edges of a future time step, it only expresses which edges must not exist.
It is multiplied element-wise onto the adjacency matrix $A$ to obtain the adjacency matrix with applied cancellations $\bar{A} = A \odot C$.

% Future Work: This would be more powerful, if there would be some different kinds of edges (social, work, education, large events, etc.)\\
% Future Work: This only takes the current time step into account but it would be desirable to look even further into the future.

\subsubsection{Test Prioritization}
When tests are limited, they should be used to discover as much as possible about the health state of the overall population.
This in turn allows to reduce the $R_0$ value in further time steps as edge cutting becomes more efficient.

Lets assume there are $t_{\text{max}}$ tests per time step.
A test reveals the true health state of an individual (ignoring false negatives and false positives)
\begin{equation}
h_{{v}_i}^{(t)} \xrightarrow{\text{test}} h_{{v}_i}^{(t+1)} \in \{\vec{e}_0, \vec{e}_1, \vec{e}_2 \}
\end{equation}

The test assignment $T$ with dimension $N$ is a binary variable describing which individuals should be tested.