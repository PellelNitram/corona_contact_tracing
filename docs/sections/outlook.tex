% !TeX root = ./main.tex
% chktex-file 46
% !TeX spellcheck = en-GB
% !TeX encoding = utf8

% - re-model current strategies: doing nothing, social distancing, isolation.

In the following we will present two extensions to the above methods.
First we demonstrate a baysian approach to estimate the populations' health state.
Second we show how the above methods could be utilized to derive disease containment policies.

\subsection{Baysian way of estimating the Probabilities}

A major drawback of our analysis is that we only take fixed disease states of individuals into account, i.e either susceptible, infected or recovered. However, in real live, due to the lack of data, these states are mostly unknown and have to be estimated by a probability distributions. If we model these probabilities based on detailed assumptions, we can use Bayes' posterior probability to update these distributions based on the individual interaction of people. Thus, we define $X$ as the random variable that represents the state of a single individual. It can be assumed, that $X$ is well approximated by the multinomial distribution ($n\gg1$):
\begin{equation}
    X = \left(X_{S}, X_{I}, X_{R}\right) \sim \text{Mn}(n;p_S^{(t)},p_I^{(t)},p_R^{(t)})
\end{equation}
where $n$ is the population size, and $p_S^{(t)}$, $p_I^{(t)}$, and $p_R^{(t)}$ are the probability for an individual to be either in susceptible, an infected or a recovered state. These probabilities change over time as the disease progresses, while

\begin{equation}
    p_S^{(t)}+p_I^{(t)}+p_R^{(t)}=1
\end{equation}

As we can only estimate the probabilities $p_S^{(t)}$, $p_I^{(t)}$, and $p_R^{(t)}$ from samples of our population. Therefore, we assume that the probabilities are taken from a prior distribution, in this case the Dirichlet distribution:

\begin{equation}
    \left(p_{S,i}^{(t)}, p_{I,i}^{(t)}, p_{R,i}^{(t)}\right) \sim \operatorname{Dir}\left(\alpha_{S,i}^{(t)}, \alpha_{I,i}^{(t)}, \alpha_{R,i}^{(t)}\right).
\end{equation}

The parameters $\alpha_{1,i}^{(0)}$, $\alpha_{2,i}^{(0)}$, and $\alpha_{3,i}^{(0)}$ can be estimated from $p_S^{(0)}$, $p_I^{(0)}$, and $p_R^{(0)}$ and are specific for each individual $i$.

If we now assume that two individuals with unknown state meet at the same location, an interaction occurs and we cannot neglect, that at least one of the individuals may carry the disease and may potentially infect the other. Therefore, we can assume, that each individual is present in an infectious superstate $\braket{x}$ where
\begin{equation}
    \braket{x} = \left(\braket{x_{S,i}}, \braket{x_{I,i}}, \braket{x_{R,i}}\right) = \left(p_S^{(t)},p_I^{(t)},p_R^{(t)}\right)
\end{equation}
If the state of an individual in this group known, due to a previous test for example, the state collapses into the right state (e.g $(0,1,0)$ for an infected individual).

Based on this state, we can define the graph based Bayesian update rules, where the infection rate $\beta$ is taken into account, as followed~\cite{rice2006mathematical}:

\begin{align}
    \alpha_{S,i}^{(t+1)\prime} &= \alpha_{1,i}^{(t)\prime}\\
    \alpha_{I,i}^{(t+1)\prime} &= \alpha_{2,i}^{(t)\prime} + \beta \cdot
    \sum_{j\in A_{v_i}}  x_{I,j}\\
    \alpha_{R,i}^{(t+1)\prime} &= \alpha_{3,i}^{(t)\prime}\\
    \alpha_{0,i}^{(t+1)\prime} &= \alpha_{S,i}^{(t+1)\prime}+\alpha_{I,i}^{(t+1)\prime}+\alpha_{R,i}^{(t+1)\prime}
\end{align}

The update is only done for the $\alpha_I$ as only the possible infection of an individual can change the infection state of another individual.

However, this model still does not take the potential recovery into account. We can add this by looking at the estimated Values of $p_{S,i}^{(t+1)}$, $p_{I,i}^{(t+1)}$, and $p_{R,i}^{(t+1)}$ 
\begin{alignat}{2}
    \operatorname{E}[p_{S,i}^{(t+1)}] &= \frac{\alpha_{S,i}^{(t+1)\prime}}{\alpha_{0,i}^{(t+1)\prime}} & &= \frac{\alpha_{S,i}^{(t+1)}}{\alpha_{0,i}^{(t+1)\prime}} \\
    \operatorname{E}[p_{I,i}^{(t+1)}]&= \frac{\alpha_{I,i}^{(t+1)\prime}}{\alpha_{0,i}^{(t+1)\prime}}-\gamma & &=\frac{\alpha_{I,i}^{(t+1)}}{\alpha_{0,i}^{(t+1)}} \\
    \operatorname{E}[p_{R,i}^{(t+1)}]&= \frac{\alpha_{R,i}^{(t+1)\prime}}{\alpha_{0,i}^{(t+1)\prime}}+\gamma & &= \frac{\alpha_{R,i}^{(t+1)}}{\alpha_{0,i}^{(t+1)}}
\end{alignat}

Based on this equation system, we can calculate $\alpha_{k,i}^{(t+1)}$ for $k\in{S,I,R}$. Needless to say, that if the expectation value of $p_{I,i}^{(t+1)}$ or $p_{R,i}^{(t+1)}$ return a value smaller than 0 or larger than 1, we assume $\gamma=0$ and therefore $\alpha_{k,i}^{(t+1)\prime}=\alpha_{k,i}^{(t+1)}$. Something similar may be found in~\cite{stojanovic2019bayesian}



\subsection{Policy Design}
In the previous sections we have shown how predict the health state of all individuals in a location tracked population.
The next step is to use this information to compute optimal policies.
These policies guide a governments' and societies' response to the spread of COVID-19 and modify how the disease is able to spread in a population.

An optimal policy always keeps the infection counts below the medical systems' capacity and the total number of infections as small as possible.
% not much real data on this.
Mathematically speaking we want to minimize the sum of all future infections
\begin{equation}\label{eq:number-of-infected}
	\min \sum_{\forall t} N^{(t)}_i
\end{equation}
whilst
\begin{equation}\label{eq:cap-constraint}
	N^{(l)}_i \leq N_{limit}, \forall t
\end{equation}

A policy may consist out of two different kinds of actions, non pharmaceutical interventions (NPI) and test prioritization (TP).


\subsubsection{Non Pharmaceutical Interventions (NPI)}
All non pharmaceutical interventions can be understood as some kind of edge removal in our graph-based approach:
\begin{itemize}
	\item Isolation of an infected individual removes all of its edges with very high probability.
	\item Quarantine of a contact person removes all of its edges with high probability.
	\item Social distancing removes some edges of many individuals.
	\item Cancellation of large events remove many edges of many individuals.
\end{itemize}

Formally speaking, the square matrix $C \in \mathbb{B}$ with dimensions $N \times N$ models desirable edge cancellations.
This is the policy, which will be optimized.
To avoid the trivial solution of $C=0$, the cancellation of all edges, we also want to minimize the the number of cancellations
\begin{equation}\label{eq:cancellations}
	\min_{C} -\sum_i \sum_j C_{ij}
\end{equation}

Note, that this matrix does not have to know the edges of a future time step, it only expresses which edges must not exist.
It is multiplied element-wise onto the adjacency matrix $A$ to obtain the adjacency matrix with applied cancellations $\bar{A} = A \odot C$.

% Future Work: This would be more powerful, if there would be some different kinds of edges (social, work, education, large events, etc.)\\
% Future Work: This only takes the current time step into account but it would be desirable to look even further into the future.

To obtain an optimal cancellation policy one thus must jointly minimize \cref{eq:number-of-infected,eq:cancellations} whilst fulfilling \cref{eq:cap-constraint}.

\subsubsection{Test Prioritization (TP)}
When tests are limited, we argue that they should be used to discover as much as possible about the health state of the overall population.
This in turn allows non pharmaceutical interventions such as school cancellations to become more efficient.
Currently there are only rough medical-based guidelines who should be tested and who should not.

Lets assume there are $t_{\text{max}}$ tests per time step.
A test reveals the true health state of an individual (ignoring false negatives and false positives)
\begin{equation}
h_{{v}_i}^{(t)} \xrightarrow{\text{test}} h_{{v}_i}^{(t+1)} \in \{\vec{e}_0, \vec{e}_1, \vec{e}_2 \}
\end{equation}

The test assignment $T$ with dimension $N$ is a binary variable describing which individuals should be tested.