% ****** Start of file apssamp.tex ******
%
%   This file is part of the APS files in the REVTeX 4.1 distribution.
%   Version 4.1r of REVTeX, August 2010
%
%   Copyright (c) 2009, 2010 The American Physical Society.
%
%   See the REVTeX 4 README file for restrictions and more information.
%
% TeX'ing this file requires that you have AMS-LaTeX 2.0 installed
% as well as the rest of the prerequisites for REVTeX 4.1
%
% See the REVTeX 4 README file
% It also requires running BibTeX. The commands are as follows:
%
%  1)  latex apssamp.tex
%  2)  bibtex apssamp
%  3)  latex apssamp.tex
%  4)  latex apssamp.tex
%
\documentclass[%
 reprint,
%superscriptaddress,
%groupedaddress,
%unsortedaddress,
%runinaddress,
%frontmatterverbose, 
%preprint,
%showpacs,preprintnumbers,
%nofootinbib,
%nobibnotes,
%bibnotes,
 amsmath,amssymb,showkeys,
 aps,
%pra,
%prb,
%rmp,
%prstab,
%prstper,
%floatfix,
]{revtex4-1}

\usepackage{graphicx}% Include figure files
\usepackage{dcolumn}% Align table columns on decimal point
\usepackage{bm}% bold math
\usepackage{hyperref}% add hypertext capabilities
\usepackage{todonotes}

\begin{document}

\title{Corona Contact Tracing: Optimal contact inhibition for decelerating the pandemic}

\author{Leonard Salewski}
\author{Thomas XXX}
\author{Matthias Blaschke}
\author{Martin Lellep}

\date{03/22/2020}

\begin{abstract}
	
	\begin{description}
		\item[Scope] Created during the \#WirVSVirus Hackathon of the German government as initiative against the spreading of the COVID-19 pandemic in early 2020.
		\item[Open source] Available under \url{https://github.com/PellelNitram/corona_contact_tracing}.
	\end{description}

	\todo[inline]{Finalise abstract as last part.}
	
	In this work we develop a graph convolutional approach to predict the health status of all agents in this simulation.	It uses past contacts as well as observed health information to derive this prediction.
	It is able to deal with partial data such as missing locations and missing health status, as it is a probabilistic approach.
	
	Lastly, this prediction can be used to derive measures to reduce the diseases' spread.
	We propose to find a optimal trade of between removing the least amount of edges in said graph (e.g.\ through quarantine, social distancing, etc.) and limiting the spread of the disease.
	% Such trade of could be found with minimal cut algorithms.
	Opposed to classical non-pharmaceutical intervention methods such as contact tracing, our approach directly identifies the nodes with the greatest potential to accelerate the diseases' spread in the network.
	
	This technical report was created within the \#WirVsVirus Hackathon of the German government and is Work in Progress!

\end{abstract}

\keywords{TODO keywords}%Use showkeys class option if keyword
                              %display desired
\maketitle

\section{\label{sec:introcution}Introduction}

% !TeX root = ./main.tex
% chktex-file 46
% !TeX spellcheck = en-GB
% !TeX encoding = utf8

The outbreak of the SARS-COV-2 virus and the associated COVID-19 illness sweep rapidly across the world. Some patients need ventilation support to survive and the exponential growth in infected persons quickly overwhelms any available medical resources~\cite{10.1001/jama.2020.2648}.

Thus the identification of contact persons is of great importance to control the spread of the disease. Some governments, such as the Singapurian, use location trace data from mobile phone providers. Other approaches are more user centric and build apps that utilise GPS data of individuals.

It is known from past outbreaks and epidemiologic research that such contact tracing and non pharmaceutical interventions (NPI) like school cancellations are important tools to reduce the impact of crisis like the ongoing one. However, both are not particularly directed interventions might come at the cost of an increased social or economic cost.

We formulate a mathematical framework that is suitable to be used for subsequent optimisations which contacts should be avoided while on the other hand decreasing the social and economic cost.

The section~\ref{sec:basics} will explain the mathematical foundations that are necessary to understand our main approach presented in section~\ref{sec:framework}. Sections~\ref{sec:working_state} and~\ref{sec:outlook} present our working state at the end of the \#WirVsVirus hackathon and provide extensive outlook, respectively. In the outlooks, additionally, we propose to use mathematical optimisation to compute very targeted NPIs (e.g.\ only cancellation of large events) and optimal placement of limited tests (e.g.\ prioritize potential super spreaders). These perspectives have the potential to control the disease with minimal effects on daily life.

\section{\label{sec:TODO}Graph-based framework}

\subsection{\label{sec:citeref}Citations and References}
A citation in text uses the command \verb+\cite{#1}+ or
\verb+\onlinecite{#1}+ and refers to an entry in the bibliography. 
An entry in the bibliography is a reference to another document.

\subsubsection{Citations}
Because REV\TeX\ uses the \verb+natbib+ package of Patrick Daly, 
the entire repertoire of commands in that package are available for your document;
see the \verb+natbib+ documentation for further details. Please note that
REV\TeX\ requires version 8.31a or later of \verb+natbib+.

\paragraph{Syntax}
The argument of \verb+\cite+ may be a single \emph{key}, 
or may consist of a comma-separated list of keys.
The citation \emph{key} may contain 
letters, numbers, the dash (-) character, or the period (.) character. 
New with natbib 8.3 is an extension to the syntax that allows for 
a star (*) form and two optional arguments on the citation key itself.
The syntax of the \verb+\cite+ command is thus (informally stated)
\begin{quotation}\flushleft\leftskip1em
\verb+\cite+ \verb+{+ \emph{key} \verb+}+, or\\
\verb+\cite+ \verb+{+ \emph{optarg+key} \verb+}+, or\\
\verb+\cite+ \verb+{+ \emph{optarg+key} \verb+,+ \emph{optarg+key}\ldots \verb+}+,
\end{quotation}\noindent
where \emph{optarg+key} signifies 
\begin{quotation}\flushleft\leftskip1em
\emph{key}, or\\
\texttt{*}\emph{key}, or\\
\texttt{[}\emph{pre}\texttt{]}\emph{key}, or\\
\texttt{[}\emph{pre}\texttt{]}\texttt{[}\emph{post}\texttt{]}\emph{key}, or even\\
\texttt{*}\texttt{[}\emph{pre}\texttt{]}\texttt{[}\emph{post}\texttt{]}\emph{key}.
\end{quotation}\noindent
where \emph{pre} and \emph{post} is whatever text you wish to place 
at the beginning and end, respectively, of the bibliographic reference
(see Ref.~[\onlinecite{witten2001}] and the two under Ref.~[\onlinecite{feyn54}]).
(Keep in mind that no automatic space or punctuation is applied.)
It is highly recommended that you put the entire \emph{pre} or \emph{post} portion 
within its own set of braces, for example: 
\verb+\cite+ \verb+{+ \texttt{[} \verb+{+\emph{text}\verb+}+\texttt{]}\emph{key}\verb+}+.
The extra set of braces will keep \LaTeX\ out of trouble if your \emph{text} contains the comma (,) character.

The star (*) modifier to the \emph{key} signifies that the reference is to be 
merged with the previous reference into a single bibliographic entry, 
a common idiom in APS and AIP articles (see below, Ref.~[\onlinecite{epr}]). 
When references are merged in this way, they are separated by a semicolon instead of 
the period (full stop) that would otherwise appear.

\paragraph{Eliding repeated information}
When a reference is merged, some of its fields may be elided: for example, 
when the author matches that of the previous reference, it is omitted. 
If both author and journal match, both are omitted.
If the journal matches, but the author does not, the journal is replaced by \emph{ibid.},
as exemplified by Ref.~[\onlinecite{epr}]. 
These rules embody common editorial practice in APS and AIP journals and will only
be in effect if the markup features of the APS and AIP Bib\TeX\ styles is employed.

\paragraph{The options of the cite command itself}
Please note that optional arguments to the \emph{key} change the reference in the bibliography, 
not the citation in the body of the document. 
For the latter, use the optional arguments of the \verb+\cite+ command itself:
\verb+\cite+ \texttt{*}\allowbreak
\texttt{[}\emph{pre-cite}\texttt{]}\allowbreak
\texttt{[}\emph{post-cite}\texttt{]}\allowbreak
\verb+{+\emph{key-list}\verb+}+.

\end{document}
%
% ****** End of file apssamp.tex ******
