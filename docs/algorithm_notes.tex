% chktex-file 46
% !TeX spellcheck = en-US
% !TeX encoding = utf8

\documentclass[]{article}

\RequirePackage{color,graphicx}
\usepackage{float}
\usepackage{longtable}
\usepackage{tabu}
\usepackage{pdfpages}
\usepackage{amsmath}
\usepackage{amsfonts}
\usepackage{amssymb}
\usepackage{csquotes}
\usepackage[american]{babel}
\usepackage[babel=true, protrusion=alltext-nott, final]{microtype}
\usepackage[nodayofweek]{datetime}
\usepackage[inline]{enumitem}
\usepackage{url}

%opening
\title{Corona Contact Tracing}
\author{TODO}

\newcommand{\indexOfState}[1]{\texttt{indOfState}(#1)}

\begin{document}

\maketitle

\begin{abstract}
	This technical report accompanies the implementation of the corona\_contact\_tracing package found at \url{https://github.com/PellelNitram/corona_contact_tracing}.

	It extends an existing stochastic SIR simulation to generate data sets suitable to simulating the current corona pandemic.
	This simulation has full access to all agents' health status.
	To mirror a real world scenario the health status of agents is only partially observed through tests.

	Next we develop a graph convolutional approach to predict the health status of all agents in this simulation.
	It uses past contacts as well as oberved health information to derive this prediction.
	It is able to deal with partial data such as missing locations and missing health status, as it is a probabilistic approach.
	
	Lastly, this prediction can be used to derive measures to reduce the diseases' spread.
	We propose to find a optimal trade of between removing the least amount of edges in said graph (e.g. through quarantine, social distancing, etc.) and limiting the spread of the disease.
	% Such trade of could be found with minimal cut algorithms.
	Opposed to classical non-pharmaceutical intervention methods such as contact tracing, our approach directly identifies the nodes with the greatest potential to accelerate the diseases' spread in the network.
	
	This package was created within the \#WirVsVirus Hackathon of the German government and is Work in Progress!
\end{abstract}

\section{Algorithm notes}

\begin{itemize}
	\item Graph convolution:
	\begin{equation}
		h_{v_i}^{(l+1)} = softmax(\sum_{j\in \hat{A}(i)} h_{v_j}^{(l)})
	\end{equation}
	with $\hat{A}(i)$ the infection-weighted neighbours of node $v_i$ as derived from the infection-weighted neighbours adjacency matrix $\hat{A}$.
	\item $\hat{A}$ is constructed from $A$ and $I$ which are the regular adjacency matrix and the infection matrix, respectively.
	\begin{itemize}
		\item The adjacency matrix $A$ is time dependent, $A(t)$, and inferred from data. In our use case, $A_{ij}=1$ if nodes $v_i$ and $v_j$, hence persons $i$ and $j$, have been in contact. This corresponds to $dist(v_i, v_j) \le \epsilon$ with $\epsilon=0$ in the discrete case that we consider here.
		\item The infection matrix is constructed as
		\begin{equation}
			I =
			\begin{pmatrix}
				0 & 0 & p_a & 0 \\
				0 & 0 & p_a & 0 \\
				0 & 0 &  0  & 0 \\
				0 & 0 &  0  & 0
			\end{pmatrix}
		\end{equation}
		with $I_{ij}$ and $i$ is the index of the host state and $j$ is the index of the contact person state. The states that we consider here are ordered as follows: unknown, susceptible, infected, recovered. $p_a$ denotes the probability of infection  after contact (also known as attack rate).
		\item $\hat{A}$, with $\hat{A}_{ij}\in \{0, 1\}$, is the adjacency matrix that takes the infection interactions into account and is computed as follows
		\begin{equation}
			\hat{A}_{ij} = A_{ij}\cdot \frac{I_{\indexOfState{i}, \indexOfState{j}} + I_{\indexOfState{j}, \indexOfState{i}}}{p_a}
		\end{equation}
		with $\indexOfState{k}$ as index of the state of agent $k$ in order to access the elements from $I$. The sum comes from the fact that both, agent $i$ and $j$, can act as host during a contact. The division by $p_a$ normalises the factor to one to ensure $\hat{A}_{ij} \in \{0, 1\}$. Since $I$ is not symmetric, $p_a$ is a proper normalisation because the sum is in $\{0, p_a\}$.
	\end{itemize}
	\item The feature matrix, $H(t)$, consists of all agents' features at time $t$ and is thereby of dimension $N\times D$ where there are $N$ agents in the population and each agent is described by $D$ features. A four dimensional feature space is used, $D=4$. The unit vectors of this space are interpreted as following:
	\begin{itemize}
		\item $\vec{e}_1$: unknown state
		\item $\vec{e}_2$: susceptible state
		\item $\vec{e}_3$: infected state
		\item $\vec{e}_4$: recovered state
	\end{itemize}
	\item Open modeling aspects:
	\begin{itemize}
		\item Incorporate that an infected person recovers over time.
	\end{itemize}
\end{itemize}

\end{document}
