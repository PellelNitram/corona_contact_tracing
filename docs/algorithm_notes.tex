% !TeX root = ./algorithm_notes.tex
% chktex-file 46
% !TeX spellcheck = en-US
% !TeX encoding = utf8


\documentclass[]{article}

\RequirePackage{color,graphicx}
\usepackage{float}
\usepackage{longtable}
\usepackage{tabu}
\usepackage{pdfpages}
\usepackage{amsmath}
\usepackage{amsfonts}
\usepackage{amssymb}
\usepackage{mathtools}
\usepackage{csquotes}
\usepackage[american]{babel}
\usepackage[babel=true, protrusion=alltext-nott, final]{microtype}
\usepackage[nodayofweek]{datetime}
\usepackage[inline]{enumitem}
\usepackage{url}

%opening
\title{Corona Contact Tracing}
\author{TODO}

\newcommand{\indexOfState}[1]{\texttt{indOfState}(#1)}
\DeclarePairedDelimiter\abs{\lvert}{\rvert}%
\DeclarePairedDelimiter\norm{\lVert}{\rVert}%

\begin{document}

\maketitle

\begin{abstract}
	This technical report accompanies the implementation of the corona\_-contact\_tracing package found at \url{https://github.com/PellelNitram/corona_contact_tracing}.

	It extends an existing stochastic SIR simulation to generate data sets suitable to simulating the current corona pandemic.
	This simulation has full access to all agents' health status.
	To mirror a real world scenario the health status of agents is only partially observed through tests.

	Next we develop a graph convolutional approach to predict the health status of all agents in this simulation.
	It uses past contacts as well as oberved health information to derive this prediction.
	It is able to deal with partial data such as missing locations and missing health status, as it is a probabilistic approach.
	
	Lastly, this prediction can be used to derive measures to reduce the diseases' spread.
	We propose to find a optimal trade of between removing the least amount of edges in said graph (e.g. through quarantine, social distancing, etc.) and limiting the spread of the disease.
	% Such trade of could be found with minimal cut algorithms.
	Opposed to classical non-pharmaceutical intervention methods such as contact tracing, our approach directly identifies the nodes with the greatest potential to accelerate the diseases' spread in the network.
	
	This technical report was created within the \#WirVsVirus Hackathon of the German government and is Work in Progress!
\end{abstract}

\section{Algorithm notes}

\subsection{SIR Model}

\subsection{Graph Convolution}
We think of partially infected population as a graph, where each individual (or agent) is a node.
Edges of this graph model contacts of two agents.
As the health state of the entire population is unknown, we use graph convolution, as explained below, to update our assumptions on the current health state of all individuals.

\begin{itemize}
	\item Graph convolution:
	\begin{equation}
		h_{v_i}^{(l+1)} = softmax(\sum_{j\in \hat{A}(i)} h_{v_j}^{(l)})
	\end{equation}
	with $\hat{A}(i)$ the infection-weighted neighbors of node $v_i$ as derived from the infection-weighted neighbors adjacency matrix $\hat{A}$.
	\item $\hat{A}$ is constructed from $A$ and $I$ which are the regular adjacency matrix and the infection matrix, respectively.
	\begin{itemize}
		\item The adjacency matrix $A$ is time dependent, $A^{(t)}$, and inferred from data. In our use case, $A_{ij}=1$ if nodes $v_i$ and $v_j$, hence persons $i$ and $j$, have been in contact. This corresponds to $dist(v_i, v_j) \le \epsilon$ with $\epsilon=0$ in the discrete case that we consider here.
		\item The infection matrix is constructed as
		\begin{equation}
			I =
			\begin{pmatrix}
				0 & p_a & 0 \\
				0 &  0  & 0 \\
				0 & p_r & 0
			\end{pmatrix}
		\end{equation}
		with $I_{ij}$ and $i$ is the index of the host state and $j$ is the index of the contact person state. The states that we consider here are ordered as follows: susceptible, infected, recovered. $p_a$ denotes the probability of infection  after contact (also known as attack rate). $p_r = 0$ models the probability of being reinfected.
		\item $\hat{A}$, with $\hat{A}_{ij}\in \{0, 1\}$, is the adjacency matrix that takes the infection interactions into account and is computed as follows
		\begin{equation}
			\hat{A}_{ij} = A_{ij}\cdot \frac{I_{\indexOfState{i}, \indexOfState{j}} + I_{\indexOfState{j}, \indexOfState{i}}}{p_a}
		\end{equation}
		with $\indexOfState{k}$ as index of the state of agent $k$ in order to access the elements from $I$. The sum comes from the fact that both, agent $i$ and $j$, can act as host during a contact. The division by $p_a$ normalises the factor to one to ensure $\hat{A}_{ij} \in \{0, 1\}$. Since $I$ is not symmetric, $p_a$ is a proper normalization because the sum is in $\{0, p_a\}$.
	\end{itemize}
	\item The feature matrix, $H^{(t)}$, consists of all agents' features at time $t$ and is thereby of dimension $N\times D$ where there are $N$ agents in the population and each agent is described by $D$ features. A three dimensional feature space is used, $D=3$, modelling the three possible health states. The unit vectors of this space are interpreted as following:
	\begin{itemize}
		\item $\vec{e}_0$: susceptible state
		\item $\vec{e}_1$: infected state
		\item $\vec{e}_2$: recovered state
	\end{itemize}
	A uniform distribution over these possible states expresses complete uncertainty of the health state of an agent.
	\item The transition of a persons' health state $H^{(t)}$ is determined by the following assumptions:
	\begin{itemize}
		\item A susceptible person always stays susceptible (unless their is a vaccine, which we do not model)
		\item A infected person has a probability $\gamma$ called recovery rate to recover
		\item A recovered person could habe a probability to be re-infected, but we assume this to be zero. Thus a recovered person always stays recovered.
	\end{itemize}
	Thus or transition matrix $T$ is:
	\begin{equation}
		T = 
		\begin{pmatrix}
			1 &     0    & 0 \\
			0 & 1-\gamma & \gamma \\
			0 &     0    & 1 \\
		\end{pmatrix}
	\end{equation}
	The update rule thus becomes:
	\begin{equation}
		H^{(t+1)} = H^{(t)} T
	\end{equation}

	\item TODO: Open modeling aspects:
	\begin{itemize}
		\item Normalize H to 1 (softmax)
		 
	\end{itemize}
\end{itemize}

\subsection{Edge Cutting Trade Offs}
All non pharmaceutical interventions (NPI) can be understood as some kind of edge removal in our graph-based approach:
\begin{itemize}
	\item Isolation of an infected individual removes all of its edges with very high probability.
	\item Quarantine of a contact person removes all of its edges with high probability.
	\item Social distancing removes some edges of of many individuals.
	\item Cancellation of large events remove many edges of many individuals.
\end{itemize}
In the following we discuss different techniques to calculate the minimal set of edge removals which reduces the $R_0$ value of the pandemic below $1$.
$R_0$ models how many people are infected by a single infected person.
It is crucial to constrain $R_0$ below zero to avoid exponential growth in the number of infected individuals.
$R_0$ can be derived from the ratio of $H^{(t+1)}$ and $H^{(t)}$:
\begin{equation}
	R_0 = \frac{H^{(t+1)}}{H^{(t)}}
\end{equation}

The square matrix $C$ with dimensions $N \times N$ models desirable edge cancellations.
It is multiplied elementwise onto the adjacency matrix $A$, thus $\bar{A} = A \odot C$ describes a adjacency matrix with applied cancellations.

To optimally limit the spread of the disease, we seek to minimize the number of cancellations $\norm{C}_0$, given that $R_0$ is below zero:
\begin{equation}
	\min_{C} \norm{C}_0\text{, s.t. }R_0 < 1
\end{equation}

TODO: This would be more powerful, if there would be some different kinds of edges (social, work, education, large events, etc.)\\
TODO: This only takes the current time step into account but it would be desirable to look even further into the future.

\end{document}
