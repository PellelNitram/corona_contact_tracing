\documentclass[]{article}

\usepackage{amsmath}
\usepackage{amssymb}

%opening
\title{TODO}
\author{TODO}

\newcommand{\indexOfState}[1]{\texttt{indOfState}(#1)}

\begin{document}

\maketitle

\begin{abstract}

\end{abstract}

\section{Algorithm notes}

\begin{itemize}
	\item Graph convolution:
	\begin{equation}
		h_{v_i}^{(l+1)} = softmax(\sum_{j\in \hat{A}(i)} h_{v_j}^{(l)})
	\end{equation}
	with $\hat{A}(i)$ the infection-weighted neighbours of node $v_i$ as derived from the infection-weighted neighbours adjacency matrix $\hat{A}$.
	\item $\hat{A}$ is constructed from $A$ and $I$ which are the regular adjacency matrix and the infection matrix, respectively.
	\begin{itemize}
		\item The adjacency matrix $A$ is time dependent, $A(t)$, and inferred from data. In our use case, $A_{ij}=1$ if nodes $v_i$ and $v_j$, hence persons $i$ and $j$, have been in contact. This corresponds to $dist(v_i, v_j) \le \epsilon$ with $\epsilon=0$ in the discrete case that we consider here.
		\item The infection matrix is constructed as
		\begin{equation}
			I =
			\begin{pmatrix}
				0 & 0 & p_a & 0 \\
				0 & 0 & p_a & 0 \\
				0 & 0 &  0  & 0 \\
				0 & 0 &  0  & 0
			\end{pmatrix}
		\end{equation}
		with $I_{ij}$ and $i$ is the index of the host state and $j$ is the index of the contact person state. The states that we consider here are ordered as follows: unknown, susceptible, infected, recovered. $p_a$ denotes the probability of infection  after contact (also known as attack rate).
		\item $\hat{A}$, with $\hat{A}_{ij}\in \{0, 1\}$, is the adjacency matrix that takes the infection interactions into account and is computed as follows
		\begin{equation}
			\hat{A}_{ij} = A_{ij}\cdot \frac{I_{\indexOfState{i}, \indexOfState{j}} + I_{\indexOfState{j}, \indexOfState{i}}}{p_a}
		\end{equation}
		with $\indexOfState{k}$ as index of the state of agent $k$ in order to access the elements from $I$. The sum comes from the fact that both, agent $i$ and $j$, can act as host during a contact. The division by $p_a$ normalises the factor to one to ensure $\hat{A}_{ij} \in \{0, 1\}$. Since $I$ is not symmetric, $p_a$ is a proper normalisation because the sum is in $\{0, p_a\}$.
	\end{itemize}
	\item The feature matrix, $H$, consists of all agents' features and is thereby of dimension $N\times D$ where there are $N$ agents in the population and each agent is described by $D$ features. A four dimensional feature space is used, $D=4$. The unit vectors of this space are interpreted as following:
	\begin{itemize}
		\item $\vec{e}_1$: unknown state
		\item $\vec{e}_2$: susceptible state
		\item $\vec{e}_3$: infected state
		\item $\vec{e}_4$: recovered state
	\end{itemize}
	\item Open modeling aspects:
	\begin{itemize}
		\item Incorporate that an infected person recovers over time.
	\end{itemize}
\end{itemize}

\end{document}
