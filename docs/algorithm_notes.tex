% chktex-file 46
% !TeX spellcheck = en-US
% !TeX encoding = utf8

\documentclass[]{article}

\RequirePackage{color,graphicx}
\usepackage{float}
\usepackage{longtable}
\usepackage{tabu}
\usepackage{pdfpages}
\usepackage{amsmath}
\usepackage{amsfonts}
\usepackage{amssymb}
\usepackage{csquotes}
\usepackage[american]{babel}
\usepackage[babel=true, protrusion=alltext-nott, final]{microtype}
\usepackage[nodayofweek]{datetime}
\usepackage[inline]{enumitem}
\usepackage{url}

%opening
\title{Corona Contact Tracing}
\author{TODO}

\begin{document}

\maketitle

\begin{abstract}
	This technical report accompanies the implementation of the corona\_contact\_tracing package found at \url{https://github.com/PellelNitram/corona_contact_tracing}.

	It extends an existing stochastic SIR simulation to generate data sets suitable to simulating the current corona pandemic.
	This simulation has full access to all agents' health status.
	To mirror a real world scenario the health status of agents is only partially observed through tests.

	Next we develop a graph convolutional approach to predict the health status of all agents in this simulation.
	It uses past contacts as well as oberved health information to derive this prediction.
	It is able to deal with partial data such as missing locations and missing health status, as it is a probabilistic approach.
	
	Lastly, this prediction can be used to derive measures to reduce the diseases' spread.
	We propose to find a optimal trade of between removing the least amount of edges in said graph (e.g. through quarantine, social distancing, etc.) and limiting the spread of the disease.
	% Such trade of could be found with minimal cut algorithms.
	Opposed to classical non-pharmaceutical intervention methods such as contact tracing, our approach directly identifies the nodes with the greatest potential to accelerate the diseases' spread in the network.
	
	This package was created within the \#WirVsVirus Hackathon of the German government and is Work in Progress!
\end{abstract}

\section{Algorithm notes}

\begin{itemize}
	\item Graph convolution:
	\begin{equation}
		h_{v_i}^{(l+1)} = softmax(\sum_{j\in N_i} h_{v_j}^{(l)})
	\end{equation}
	with $N_i$ the infection-weighted neighbours of node $v_i$.
	\item From $A$ and $I$ where $I$ is the infection matrix.
	\begin{itemize}
		\item The infection matrix as
		\begin{equation}
			I =
			\begin{pmatrix}
				0 & 0 & p_a & 0 \\
				0 & 0 & p_a & 0 \\
				0 & 0 &  0  & 0 \\
				0 & 0 &  0  & 0
			\end{pmatrix}
		\end{equation}
		with $I_{ij}$ so that $i$ corresponds to host and $j$ to contact person and ordering unknown, susceptible, infected, recovered. $p_a$ denotes the probability of infection (also known as attack rate) after contact.
		\item The adjacency matrix $A$ is time dependent, $A(t)$, and inferred from data. In our use case, $A_{ij}=1$ if nodes $v_i$ and $v_j$, hence persons $i$ and $j$, have been in contact. This corresponds to $dist(v_i, v_j) \le \epsilon$ with $\epsilon=0$ in the discrete case that we consider here.
		\item $\hat{A}$ is the adjacency matrix that takes tie infection interactions into account and is computed as follows: $\hat{A}_{ij} = A_{ij}\cdot I_{}$.
	\end{itemize}
	\item H matrix dfn
	\item 
\end{itemize}

\end{document}
