\documentclass[]{article}

\usepackage{amsmath}
\usepackage{amssymb}

%opening
\title{TODO}
\author{TODO}

\begin{document}

\maketitle

\begin{abstract}

\end{abstract}

\section{Algorithm notes}

\begin{itemize}
	\item Graph convolution:
	\begin{equation}
		h_{v_i}^{(l+1)} = softmax(\sum_{j\in N_i} h_{v_j}^{(l)})
	\end{equation}
	with $N_i$ the infection-weighted neighbours of node $v_i$.
	\item From $A$ and $I$ where $I$ is the infection matrix.
	\begin{itemize}
		\item The infection matrix as
		\begin{equation}
			I =
			\begin{pmatrix}
				0 & 0 & p_a & 0 \\
				0 & 0 & p_a & 0 \\
				0 & 0 &  0  & 0 \\
				0 & 0 &  0  & 0
			\end{pmatrix}
		\end{equation}
		with $I_{ij}$ so that $i$ corresponds to host and $j$ to contact person and ordering unknown, susceptible, infected, recovered. $p_a$ denotes the probability of infection (also known as attack rate) after contact.
		\item The adjacency matrix $A$ is time dependent, $A(t)$, and inferred from data. In our use case, $A_{ij}=1$ if nodes $v_i$ and $v_j$, hence persons $i$ and $j$, have been in contact. This corresponds to $dist(v_i, v_j) \le \epsilon$ with $\epsilon=0$ in the discrete case that we consider here.
		\item $\hat{A}$ is the adjacency matrix that takes tie infection interactions into account and is computed as follows: $\hat{A}_{ij} = A_{ij}\cdot I_{}$.
	\end{itemize}
	\item H matrix dfn
	\item 
\end{itemize}

\end{document}
